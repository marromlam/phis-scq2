\documentclass[aspectratio=43]{beamer}
\usetheme{Compostela}



%%%%%%%%%%%%%%%%%%%%%%%%%%%%%%%%%%%%%%%%%%%%%%%%%%%%%%%%%%%%%%%%%%%%%%%%%%%%%%%%
%%%%%%%%%%%%%%%%%%%%%%%%%%%% Talk Configuration %%%%%%%%%%%%%%%%%%%%%%%%%%%%%%%%

\newcommand{\TalkPlace}{$\phi_s$ meeting}
\newcommand{\TalkAuthor}{
\href{mailto:veronika.chobanova@cern.ch}{Veronika Chobanova}\\
\href{mailto:diego.martinez.santos@cern.ch}{Diego Martínez Santos}\\
\href{mailto:marcos.romero.lamas@cern.ch}{Marcos Romero Lamas}
}
\newcommand{\TalkAuthorShort}{Veronika Chobanova, Diego Martínez, Marcos Romero}
\newcommand{\TalkTitle}{Acceptances}
\newcommand{\TalkTitleShort}{Acceptances}
\newcommand{\TalkInstitute}{}
\newcommand{\TalkDate}{July 9th}
\newcommand{\TalkDateNumber}{2020/07/09}

%%%%%%%%%%%%%%%%%%%%%%%%%%%%%%%%%%%%%%%%%%%%%%%%%%%%%%%%%%%%%%%%%%%%%%%%%%%%%%%%



\usepackage{array}
\newcommand{\TupleVersion}{v0r5}

\newenvironment{variableblock}[3]{%
  \setbeamercolor{block body}{#2}
  \setbeamercolor{block title}{#3}
  \begin{block}{#1}}{\end{block}}

%%%%%%%%%%%%%%%%%%%%%%%%%%%%%%%%%%%%%%%%%%%%%%%%%%%%%%%%%%%%%%%%%%%%%%%%%%%%%%%%
\begin{document} %%%%%%%%%%%%%%%%%%%%%%%%%%%%%%%%%%%%%%%%%%%%%%%%%%%%%%%%%%%%%%%
%%%%%%%%%%%%%%%%%%%%%%%%%%%%%%%%%%%%%%%%%%%%%%%%%%%%%%%%%%%%%%%%%%%%%%%%%%%%%%%%



%%%%%%%%%%%%%%%%%%%%%%%%%%%%%%%%%%%%%%%%%%%%%%%%%%%%%%%%%%%%%%%%%%%%%%%%%%%%%%%%
%%%%%%%%%%%%%%%%%%%%%%%%%%%%% TITLE PAGE & CONTENTS %%%%%%%%%%%%%%%%%%%%%%%%%%%%
%%%%%%%%%%%%%%%%%%%%%%%%%%%%%%%%%%%%%%%%%%%%%%%%%%%%%%%%%%%%%%%%%%%%%%%%%%%%%%%%

\begin{frame}[plain,overlaytitlepage=0.9]
  \begin{minipage}[b][\textheight][b]{5cm}
    \includegraphics[height=0.5cm]{logos/igfae_bw}\hspace{1mm}
    \includegraphics[height=0.5cm]{logos/usc_bw}\hspace{1mm}
    \includegraphics[height=0.5cm]{logos/xunta_bw}\hspace{1mm}\\[2mm]
    \includegraphics[height=0.5cm]{logos/maeztu_bw}\hspace{1mm}
    \includegraphics[height=0.5cm]{logos/lhcb_bw}\\[-1mm]
  \end{minipage}
\end{frame}

\begin{frame}[plain,overlaytoc=0.9]
  \addtocounter{framenumber}{-1}
  \hspace*{5.3cm}\begin{minipage}{8cm}
    \tableofcontents
  \end{minipage}
\end{frame}

%%%%%%%%%%%%%%%%%%%%%%%%%%%%%%%%%%%%%%%%%%%%%%%%%%%%%%%%%%%%%%%%%%%%%%%%%%%%%%%%




\section{News}

%%%%%%%%%%%%%%%%%%%%%%%%%%%%%%%%%%%%%%%%%%%%%%%%%%%%%%%%%%%%%%%%%%%%%%%%%%%%%%%%
\subsection{Solving the p.d.f. problem}
\begin{frame}[default]
\frametitle{noframetitle}
\framesubtitle{ \texttt{\color{scqgreen}SOLVED} }

The reason HD and SCQ were different was the computation of \texttt{faddeeva}
function, which implementation is different. Once this was solved, SCQ and HD
agree up to $10^{-14}$.

\includegraphics[width=0.45\textwidth]{gpx/wofz.pdf}
\includegraphics[width=0.45\textwidth]{gpx/wofz_differences.pdf}

Picture on the right shows comparison against Mathematica.\\

\vspace*{1cm}
\footnotesize
As a matter of fact
ROOT precision is
$1.870\times10^{-7} + i 8.257\times 10^{-10}$\\
while SciPy (best one) is
$2.046\times10^{-14} + i 2.976 \times10^{-13}$.

\end{frame}
%%%%%%%%%%%%%%%%%%%%%%%%%%%%%%%%%%%%%%%%%%%%%%%%%%%%%%%%%%%%%%%%%%%%%%%%%%%%%%%%



%%%%%%%%%%%%%%%%%%%%%%%%%%%%%%%%%%%%%%%%%%%%%%%%%%%%%%%%%%%%%%%%%%%%%%%%%%%%%%%%
\begin{frame}[default]
\frametitle{noframetitle}

\begin{columns}
  \begin{column}{0.48\textwidth}
    \begin{variableblock}{Complementary error function core}{bg=scqindigo!20}{bg=scqindigo}
      \textbf{function} \textsc{cerfc\_core} $(z)$\\
      \hspace{0.25cm}$a \leftarrow \text{exp}(z^2)$\\
      \hspace{0.25cm}\textbf{if} $\Re (z) >= 0$ \textbf{then}\\
      \hspace{0.50cm}\textbf{return} $a \cdot \text{faddeva}(-\Im(z)+\Re(z)i)$\\
      \hspace{0.25cm}\textbf{else}\\
      \hspace{0.50cm} $b \leftarrow a \cdot \text{faddeva}(\Im(z)-\Re(z)i)$\\
      \hspace{0.50cm}\textbf{return} $2-\Re(b) + \Im(b)i$\\
      \hspace{0.25cm}\textbf{end}\\
      \textbf{end}
    \end{variableblock}

    \vspace*{6mm}
    \textsc{faddeva}  code can be found at \url{https://github.com/PyCOMPLETE/faddeevas/blob/master/cernlib_cuda/wofz.cu}

  \end{column}
  \begin{column}{0.48\textwidth}
    \begin{variableblock}{Complementary error function}{bg=scqorange!20}{bg=scqorange}
      \textbf{function} \textsc{cerfc} $(z)$\\
      \hspace{0.25cm}\textbf{if} $\Im (z) < 0$ \textbf{then}\\
      \hspace{0.50cm}\textbf{return} $2 - \text{cerf\_core}(-z)$\\
      \hspace{0.25cm}\textbf{else}\\
      \hspace{0.50cm}\textbf{return} $\text{cerf\_core}(z)$\\
      \hspace{0.25cm}\textbf{end}\\
      \textbf{end}
    \end{variableblock}

    \begin{center}
      \includegraphics[width=0.6\textwidth]{gpx/erfc_crop.pdf}
    \end{center}

  \end{column}

\end{columns}





\end{frame}
%%%%%%%%%%%%%%%%%%%%%%%%%%%%%%%%%%%%%%%%%%%%%%%%%%%%%%%%%%%%%%%%%%%%%%%%%%%%%%%%



%%%%%%%%%%%%%%%%%%%%%%%%%%%%%%%%%%%%%%%%%%%%%%%%%%%%%%%%%%%%%%%%%%%%%%%%%%%%%%%%
\begin{frame}[default]
\frametitle{noframetitle}

\begin{itemize}
  \item After all changes made to \texttt{phis-scq}, OpenCL was no longer supported.
  \item The whole code was rewritten to create a metaprogrammable kernel that runs on any platform intel CPU/GPU, (amd CPU/GPU) and nVidia GPU.
  \begin{itemize}
    \item It's way more complicated to read.
    \item It swiches some keywords depending on the platform.
    \item It's based on \texttt{reikna}, a pure Python GPGPU library.
    \item Can't be compiled with standard \texttt{gcc} or \texttt{clang}.
  \end{itemize}
  \item It's not completely finished, but the previous images were created with it.
\end{itemize}

\end{frame}
%%%%%%%%%%%%%%%%%%%%%%%%%%%%%%%%%%%%%%%%%%%%%%%%%%%%%%%%%%%%%%%%%%%%%%%%%%%%%%%%



%%%%%%%%%%%%%%%%%%%%%%%%%%%%%%%%%%%%%%%%%%%%%%%%%%%%%%%%%%%%%%%%%%%%%%%%%%%%%%%%
\subsection{Solving differences in pdf-weighting}
\begin{frame}[default]
\frametitle{noframetitle}
\framesubtitle{ \texttt{\color{scqgreen}SOLVED} }

\begin{itemize}
  \item We only faced differences in $B^0$ MC pdf-weighting.
  \item We were using different variables within the p.d.f., due to the absence of some branches in the old tuples.
  \item Finally we decided to use:
\end{itemize}
\begin{center}
  \begin{tabular}{r|ll}
                           & true time & true ID\\ \hline
  HD-fitter                & \texttt{time} & \texttt{X\_ID}\\
  bs2jpsiphi-hd, phis-scq  & \texttt{truetime\_GenLvl} & \texttt{B\_ID\_GenLvl}
  \end{tabular}
\end{center}

\begin{itemize}
  \item Now, we get the \textbf{same time acceptance} for all years.
\end{itemize}

\end{frame}
%%%%%%%%%%%%%%%%%%%%%%%%%%%%%%%%%%%%%%%%%%%%%%%%%%%%%%%%%%%%%%%%%%%%%%%%%%%%%%%%



%%%%%%%%%%%%%%%%%%%%%%%%%%%%%%%%%%%%%%%%%%%%%%%%%%%%%%%%%%%%%%%%%%%%%%%%%%%%%%%%
\subsection{Different uncertainty estimation}
\begin{frame}[default]
\frametitle{noframetitle}
\framesubtitle{ \texttt{\color{scqred}UNDER INVESTIGATION} }

\begin{itemize}
  \item Although we get the same value for each time acceptance coefficient, SCQ always gets lower uncertainty estimator (reading directly from \texttt{hesse}).
  \item SCQ got slightly better minima than HD.
  \item Likelihood value is the same for a given parameter-set.
\end{itemize}


\end{frame}
%%%%%%%%%%%%%%%%%%%%%%%%%%%%%%%%%%%%%%%%%%%%%%%%%%%%%%%%%%%%%%%%%%%%%%%%%%%%%%%%



%%%%%%%%%%%%%%%%%%%%%%%%%%%%%%%%%%%%%%%%%%%%%%%%%%%%%%%%%%%%%%%%%%%%%%%%%%%%%%%%
\subsection{BDT issues}
\begin{frame}[default]
\frametitle{noframetitle}
\framesubtitle{ \texttt{\color{scqred}UNDER INVESTIGATION} }

\begin{center}
  \includegraphics[width=\textwidth]{gpx/angular_pipeline.pdf}
\end{center}


\begin{itemize}
  \item Step 5 is very sensible to the BDT configuration we use. We were getting non-sense \texttt{kkpWeight}s, thus iterative procedure did not converge.
  \item We found a BDT configuration that works out, but...
\end{itemize}


\end{frame}
%%%%%%%%%%%%%%%%%%%%%%%%%%%%%%%%%%%%%%%%%%%%%%%%%%%%%%%%%%%%%%%%%%%%%%%%%%%%%%%%



%%%%%%%%%%%%%%%%%%%%%%%%%%%%%%%%%%%%%%%%%%%%%%%%%%%%%%%%%%%%%%%%%%%%%%%%%%%%%%%%
\subsection{MC combination issues}
\begin{frame}[default,fragile]
\frametitle{noframetitle}
\framesubtitle{ \texttt{\color{scqred}UNDER INVESTIGATION} }

\footnotesize

\begin{verbatim}
Value of chi2/dof = 3.37/9 corresponds to a p-value of 0.9478
Current angular weights for Bs2JpsiPhi-2016-unbiased sample are:
      MC |   MC_dG0 | Combined
+1.00000 | +1.00000 | +1.0+/-0
+1.03661 | +1.03641 | +1.03651+/-0.00070
+1.03658 | +1.03573 | +1.03620+/-0.00069
-0.00102 | -0.00083 | -0.00090+/-0.00054
+0.00042 | +0.00059 | +0.00049+/-0.00033
+0.00025 | +0.00036 | +0.00028+/-0.00033
+1.01026 | +1.00967 | +1.01000+/-0.00047
+0.00007 | -0.00012 | -0.00002+/-0.00042
+0.00061 | -0.00059 | +0.00007+/-0.00043
-0.00162 | -0.00257 | -0.00205+/-0.00089
\end{verbatim}

\normalsize

\begin{itemize}
  \item Here everything seems OK, p-value is in $[0.5, 0.96]$ as in the previous analysis
\end{itemize}


\end{frame}
%%%%%%%%%%%%%%%%%%%%%%%%%%%%%%%%%%%%%%%%%%%%%%%%%%%%%%%%%%%%%%%%%%%%%%%%%%%%%%%%



%%%%%%%%%%%%%%%%%%%%%%%%%%%%%%%%%%%%%%%%%%%%%%%%%%%%%%%%%%%%%%%%%%%%%%%%%%%%%%%%
\addtocounter{framenumber}{-1}
\begin{frame}[default,fragile]
\frametitle{noframetitle}
\framesubtitle{ \texttt{\color{scqred}UNDER INVESTIGATION} }

\footnotesize

\begin{verbatim}
Value of chi2/dof = 45.32/9 corresponds to a p-value of 8.056e-07
Current angular weights for Bs2JpsiPhi-2018-unbiased sample are:
      MC |   MC_dG0 | Combined
+1.00000 | +1.00000 | +1.0+/-0
+1.03013 | +1.02329 | +1.0282+/-0.0013
+1.02958 | +1.02783 | +1.0291+/-0.0013
+0.00032 | -0.00434 | -0.00096+/-0.00099
+0.00002 | +0.00069 | +0.00020+/-0.00061
+0.00025 | +0.00762 | +0.00219+/-0.00062
+1.00629 | +1.00599 | +1.00610+/-0.00088
-0.00023 | -0.00052 | -0.00026+/-0.00078
+0.00056 | +0.00092 | +0.00045+/-0.00080
+0.00053 | +0.00470 | +0.0016+/-0.0017
\end{verbatim}

\normalsize

\begin{itemize}
  \item Here MC and MC $\Delta\Gamma = 0$ gave very different angular weights
\end{itemize}

\end{frame}
%%%%%%%%%%%%%%%%%%%%%%%%%%%%%%%%%%%%%%%%%%%%%%%%%%%%%%%%%%%%%%%%%%%%%%%%%%%%%%%%



\input{figures}
\input{tables}



\end{document}
